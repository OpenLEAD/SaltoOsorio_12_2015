\section{Conclusão}

A operação com o robô ROSA na Usina de Salto Osório apresentou resultados
positivos. Todo o monitoramente foi realizado com sucesso e foi possível a
adaptação do sistema à viga. Alguns imprevistos devem ser relatados:

\begin{itemize}
  \item Quando o sistema foi instalado no lado direito da viga (perspectiva do
  operador), o comprimento do cabo do sensor indutivo da garra esquerda não foi
  suficiente.
  \item Três abraçadeiras metálicas não foram suficientes para deixar o sistema
  imóvel a grandes perturbações.
  \item O cabo umbilical por não possuir carretel é sempre difícil de ser
  manipulado.
  \item A eletrônica de superfície não iniciou nos primeiros testes e foi
  identificado que o computador estava preso na tela de boot, havendo a
  necessidade de pressionar algum botão do teclado para a inicialização.
  \sylvain consertou o problema na eletrônica de superfície, mas o problema pode
  ainda acontecer na eletrônica embarcada.
  \item O olhal do stoplog na Usina de Salto Osório é diferente ao stoplog de
  UHE Jirau. As garras se acoplam a cilindros no stoplog e podem correr pelo
  comprimento do cilindro, de forma que o sensor indutivo não perceba a presença
  de stoplogs mesmo quando há acoplamento.
\end{itemize}

O técnico do pórtico rolante aprovou o aplicativo e comentou a interface
intuitiva. Os alertas sonoros se destacaram, avisando ao operador inclinação da
viga acima do esperado, possibilitando reação imediata, e pesca de stoplog bem
sucedida. 
