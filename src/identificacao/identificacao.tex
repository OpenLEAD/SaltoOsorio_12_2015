%-----------------------------------------------------------------------
%
%   UFRJ  - Universidade Federal do Rio de Janeiro
%   COPPE - Coordenação dos Programas de Pós-graduação em Engenharia
%   PEE   - Programa de Engenharia Elétrica
%
%
%   Projeto EMMA - Metodologia para revestimento robótico de turbinas in situ
%
%   Identificação
%                                                         Ramon R. Costa
%                                                         07/jul/15, Rio
%-----------------------------------------------------------------------
%\section{Identificação}

\dado{Título}{
  R.O.S.A. - Robô para operações de stoplogs alagados \\
}

\dado{Proponente}{
  Universidade Federal do Rio de Janeiro (UFRJ) \\[2mm]
  Fundação Coordenação de Projetos, Pesquisas e Estudos Tecnológicos (COPPETEC) \\
}

\dado{Contratante}{
  ESBR - Energia Sustentável do Brasil S.A. \\
}

\dado{Execução}{
  Grupo de Simulação e Controle em Automação e Robótica (GSCAR) \\
}

 \dado{Contrato}{
   Jirau 151/13 \\
 }

 \dado{P\&D ANEEL}{
   6631.0002/2013 \\
 }

%\dado{COPPETEC}{
%  N.D. \\
%}

\dado{Início}{
8/10 /2013 \\

}

\dado{Prazo}{
  17 meses \\
}

\dado{Orçamento}{
  R\$ 4.364.217,78 \\
}

\dado{Coordenador}{
  Ramon Romankevicius Costa \\
}

\dado{Gerente}{
  Breno Bellinati de Carvalho \\
}

Os engenheiros que realizaram a viagem: \renan, \estevao e \sylvain. 
%---------------------------------------------------------------------
\fim